\documentclass{article}
\usepackage{hyperref}
\begin{document}
\tableofcontents
\section{Multimedia}
	\subsection{Home Cinema}
		\href{https://kodi.tv/}{Kodi} - No introduction needed, highly customizeble and widely used on mutiple platforms.\\
		\href{https://emby.media/}{Emby} - I use Emby to manage my Media library and i have the Emby Add-on on Kodi so it automatically syncs.
	\subsection{Video and Movies}
		\href{https://mpv.io/}{MPV} - Lightweight Media player with a suprising amount of configuration in the config file. Just drag and drop.\\
		\href{https://mpc-hc.org/}{MPC-HC} (Windows) - What i used before MPV, pretty good with options for filters and such if you care to install them. I prefer it over VLC but not as great as MPV.
	\subsection{Audio and Music}
		\href{https://cmus.github.io/}{Cmus} - Lightweight terminal music player that unlike a mpd player doesn't rely on a server backend.\\
		\href{https://getmusicbee.com/}{Musicbee} (Windows) - Is what i used after moving from Windows 7 looks nice by default and has mutiple skins pre-packed.\\
		\href{https://www.foobar2000.org/}{Foobar2000} (Windows) - What i used in the past, extremely customizible and extremely light on resources. May take some time to set up according to your preferences.
\section{Torrenting}
	\href{https://www.qbittorrent.org/}{qBittorrent} - My main Torrenting program, but everything else here is also reccomended.\\
	\href{https://deluge-torrent.org/}{Deluge}\\
	\href{https://transmissionbt.com/}{Transmission} - Used to be Linux only but recently got a Windows version.
\section{Password Manager}
	\href{https://keepassxc.org/}{Keepass XC} - Also has windows version.\\
	\href{https://keepass.info/}{Keepass 2} (Windows) - Able to be run on Linux with mono.\\
	\href{https://play.google.com/store/apps/details?id=keepass2android.keepass2android}{Keepass2Android} (Android)
\section{Screenshot Program}
	\href{https://github.com/Francesco149/sharenix}{ShareNIX} - Probably the best Linux screenshot program ever. Just type in the command or bind the command to a key combination, select an area and it will instantly upload it and copy the link for you to paste.\\
	\href{https://getsharex.com/}{ShareX} (Windows) - Pretty customizable screenshot program. Also has the option to record .gifs but i never got it working when i was using it.
\section{Image Viewer}
	\href{https://github.com/muennich/sxiv}{SXIV} - Simple and fast image viewer with the option to pipeline image locations.\\
	\href{http://www.bandisoft.com/honeyview/}{Honeyview} (Windows) - Fast image viewer with nice options for customization. If you are using the default windows image viewer then you're a cuck and you should get either Honeyview or a different fast image viewer.
\section{Utilities}
	\href{https://fishshell.com/}{Fish} - Bash replacement with auto-suggestions and other helpfull features. Don't replace bash with it just have fish auto launch whenever you open your terminal emulator. Having fish in your bashrc file can also work but it can have issues.\\
	\href{https://github.com/dylanaraps/pywal}{Pywal} - Amazing program that changes the colourscheme of your computer based on the background. With a bit of work you can make it change the colours of almost every program you use on a daily basis.\\
	\href{https://github.com/dylanaraps/neofetch}{Neofetch} - Displays your system information in a nice summary + an ascii art picture of your distrobutions logo. Has the option to display a custom picture.\\
	\href{https://www.voidtools.com/}{Everything} (Windows) - Program to search your files, way faster than any stock windows search engine. Handy if you need to find a file system wide. Helped me find stuff that i thought was lost forever.\\
	\href{https://chocolatey.org/}{Chocolatey} (Windows) - Package manager for windows. Install, update and manage your applications the same way you would on Linux. Haven't tried it out yet myself but looks pretty damn promising.\\
	\href{https://dban.org/}{DBAN (Darik's Boot and Nuke)} - Even if you delete everything off a drive, some stuff can still be recovered using disk recovery software. If you're ever selling a old harddrive it's a good idea to completly wipe everything. Burn the ISO to a USB and boot into it, select DoD for method and select the amount of passes. One pass is fine. Three if you're paranoid or have particularly sensitive data.




\end{document}
